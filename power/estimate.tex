\documentclass[../main.tex]{subfiles}

\begin{document}

\chapter{Power Consumption Estimate}

\section{Power Dissipating Components}

\begin{itemize}
    %% \setlength{\itemsep}{0pt} \setlength{\parskip}{0pt}
    \item Four Solenoid Valves
    \item One Esp 32
    \item One DHT 22 sensor
    \item One 12V pump
    \item Two 2-channel 12V Relay
\end{itemize}

\section{A Duty Cycle}

\subsection{Assumptions}

There will be a total of \textbf{20 plants}, and each plant needs \textbf{200 mL} a day.

$\therefore$ total of \textbf{200 mL x 20 = 4000 mL}, of water is needed each day.

$\therefore$ \textbf{40 L} will last for \textbf{Approx. 10 days}.

\subsection{Duty Cycle}

A single Duty Cycle consist of:

\begin{itemize}
    %% \setlength{\itemsep}{0pt} \setlength{\parskip}{0pt}
    \item Filling the Tank to \textbf{Full}.
    \item Watering the Plants Daily for \textbf{10 Days}.
\end{itemize}

\section{Voltage and Current Rating for Components}

\subsection{Esp 32}

\begin{itemize}
    \setlength{\itemsep}{0pt} \setlength{\parskip}{0pt}
    \item Working Voltage = $3.3$ V
\end{itemize}

\begin{center}
    \begin{tabularx} {\textwidth} {
            >{\raggedright\arraybackslash\hsize=0.3\hsize}X
            >{\raggedright\arraybackslash\hsize=0.55\hsize}X
            >{\raggedleft\arraybackslash\hsize=0.25\hsize}X
            >{\raggedleft\arraybackslash\hsize=0.15\hsize}X
        }
        \toprule
        {\bfseries Mode} & {\bfseries Description} & {\bfseries Current Draw} & {\bfseries Unit} \\
        \midrule
        Active Mode I & Basically everything is active including Wifi and Bluetooth. Esp is acting as a WiFi Hotspot & 240 & mA \\
        Active Mode II & Basically everything is active including Wifi and Bluetooth. Esp is acting as a WiFi Reciver & 160 & mA \\
        Modem Sleep Mode & Wifi and Bluetooth are inactive and everything else is active. & 20 & mA \\
        Sleep Mode & Esp Core Processor is inactive, RAM is active and ULP is active. & 1 & mA \\
        \bottomrule
    \end{tabularx}
    \captionof{table}{Esp 32 Modes and respective current consumption.}
    \label{tbl:espModes}
\end{center}

NOTE: There are \textbf{Deep Sleep Mode} and \textbf{Hibernation Mode}. We might not be needing them.

\subsection{DHT 22 Sensor}

\begin{itemize}
    \item Working Voltage = $3.3 - 6$ V
\end{itemize}

\begin{center}
    \begin{tabularx} {\textwidth} {
            >{\raggedright\arraybackslash\hsize=0.2\hsize}X
            >{\raggedright\arraybackslash\hsize=0.25\hsize}X
            >{\raggedright\arraybackslash\hsize=0.2\hsize}X
            >{\raggedright\arraybackslash\hsize=0.65\hsize}X
        }
        \toprule
        {\bfseries Mode} & {\bfseries Current Draw} & {\bfseries Unit} & {\bfseries Remark} \\
        \midrule
        Idle & 0.5 & mA & If we dettach the sensor from the circuit using a MFET, we can decrease the current draw to 0mA \\
        Active & 1 - 2 & mA & - \\
        \bottomrule
    \end{tabularx}
    \captionof{table}{Current Draw of DHT 22 Sensor in different Modes.}
    \label{tbl:dhtCurrent}
\end{center}

\subsection{Solenoid Valve}

\begin{itemize}
    \item Working Voltage = $12$ V
\end{itemize}

\begin{center}
    \begin{tabularx} {\textwidth} {
            >{\raggedright\arraybackslash\hsize=0.2\hsize}X
            >{\raggedright\arraybackslash\hsize=0.25\hsize}X
            >{\raggedright\arraybackslash\hsize=0.2\hsize}X
            >{\raggedright\arraybackslash\hsize=0.65\hsize}X
        }
        \toprule
        {\bfseries Mode} & {\bfseries Current Draw} & {\bfseries Unit} & {\bfseries Remark} \\
        \midrule
        Idle & 0 & mA & Solenoid will be dettached from the circuit using the Relay while we are not using it. \\
        Active & 320 - 600 & mA & - \\
        \bottomrule
    \end{tabularx}
    \captionof{table}{Current Draw of Solenoid Valve in different Modes.}
    \label{tbl:solenoidCurrent}
\end{center}

\subsection{Hydraulic Pump (R365)}

\begin{itemize}
    \item Working Voltage = $8 - 12$ V
    \item Maximum Flow = $2 - 3$ L/m
    \item Maximum Outlet Pressure = $1 - 2.5$ kg
\end{itemize}

\begin{center}
    \begin{tabularx} {\textwidth} {
            >{\raggedright\arraybackslash\hsize=0.2\hsize}X
            >{\raggedright\arraybackslash\hsize=0.25\hsize}X
            >{\raggedright\arraybackslash\hsize=0.2\hsize}X
            >{\raggedright\arraybackslash\hsize=0.65\hsize}X
        }
        \toprule
        {\bfseries Mode} & {\bfseries Current Draw} & {\bfseries Unit} & {\bfseries Remark} \\
        \midrule
        Idle & 0 & mA & Pump will be dettached from the circuit using the Relay while we are not using it. \\
        No Load & 230 & mA & - \\
        On Load & 500 - 700 & mA & If we were using a brushless one, this would have been ~320mA, but it costs more, almost 500/-. \\
        \bottomrule
    \end{tabularx}
    \captionof{table}{Current Draw of Hydraulic pump in different Modes.}
    \label{tbl:pumpCurrent}
\end{center}

\subsection{2 Channel Relay Board}

\begin{itemize}
    \item Trigger Voltage = $12$ V
\end{itemize}

\begin{center}
    \begin{tabularx} {\textwidth} {
            >{\raggedright\arraybackslash\hsize=0.2\hsize}X
            >{\raggedright\arraybackslash\hsize=0.25\hsize}X
            >{\raggedright\arraybackslash\hsize=0.2\hsize}X
            >{\raggedright\arraybackslash\hsize=0.65\hsize}X
        }
        \toprule
        {\bfseries Mode} & {\bfseries Current Draw} & {\bfseries Unit} & {\bfseries Remark} \\
        \midrule
        Idle & 2 & mA & If we dettach the Relay module from the circuit using a MFET, we can decrease the current draw to 0mA \\
        Triggerd & 30 - 50 & mA & - \\
        \bottomrule
    \end{tabularx}
    \captionof{table}{Current Draw of 2 Channel Relay Board in different Modes.}
    \label{tbl:relayCurrent}
\end{center}

NOTE: This is for a single Relay Board.

\section{Power Consumption during a Single Duty Cycle}

Power Consumption during a single Duty Cycle can be split into two parts:

\begin{itemize}
    \item Power Consumed during \textbf{Pumping the Water into the 40L Tank}.
    \item Power Consumed during \textbf{Doing everything else Daily for the 10 Days}.
\end{itemize}

\begin{figure}[h!]
    \centering
    \resizebox{0.8\textwidth}{!}{
        \includegraphics{./tikzpics/epicpowerconsumptionblock.pdf}
    }
\end{figure}

\subsection{Duration of Pumping Water into the 40L Tank}

This is need to be done once during a Duty Cycle.
Let the minimum pump flow be \textbf{1 L/m} and $T$ be the time taken to fill the Tank,

\begin{align*}
    \therefore \: T &= \dfrac{40}{1} \times 60 \\
    & = 2,400 \: seconds
\end{align*}

\subsection{Per Day Power Consumption}

Per Day Power Consumption can be furthur divided into these three:

\begin{itemize}
    \item Power Consumed during \textbf{Standby}.
    \item Power Consumed while \textbf{Sensing the Temperature and Humidity}.
    \item Power Consumed during \textbf{Actually watering the plants}.
\end{itemize}

\subsubsection{Duration of Standby}

The total Standby time for a single day will be:

\begin{align*}
    \mbox{The total duration Standby} &= 24 \times 60 \times 60 \\
    &= 86,400 \: seconds
\end{align*}

\subsubsection{Duration of Sensing}

For a day, we will be sensing the temperature and humidity for \textbf{10 seconds},
during every \textbf{30 minute} interval.

\begin{align*}
    \mbox{The total duration of Sensing} &= 24 \times 2 \times 10 \\
    &= 480 \: seconds
\end{align*}

\subsubsection{Duration of Watering Plants}

\indent Only one of the total 4 Solenoids will be open at a time. We will be watering
each row for a period of \textbf{10 seconds} and all the rows will be watered for
\textbf{4 times} a day.

\begin{align*}
    \mbox{The total duration of Watering} &= 4 \times 4 \times 10 \\
    &= 160 \: seconds
\end{align*}

\subsection{Worst Case Scenario}

\subsubsection{Assumptions}

In the worst case scenario, we are assuming these things to be true:

\begin{itemize}
    \item Esp 32 is in \textbf{Active Mode I}, while we are doing something, and in \textbf{Modem Sleep Mode}, while Standby.
\end{itemize}

\subsubsection{Pumping Water}

\begin{center}
    \begin{tabularx} {\textwidth} {
            >{\raggedright\arraybackslash\hsize=0.1\hsize}X
            >{\raggedright\arraybackslash\hsize=0.2\hsize}X
            *{4}{>{\centering\arraybackslash\hsize=0.125\hsize}X}
            >{\centering\arraybackslash\hsize=0.2\hsize}X
        }
        \toprule
        & {\bfseries Active Components} & {\bfseries Working Voltage (V)}
        & {\bfseries Current Draw (A)} & {\bfseries Power (W)}
        & {\bfseries Time Duration (s)} & {\bfseries Power x Time (W.s)} \\
        \midrule
        & Esp 32 & 3.3 V & 0.24 A & 0.792 W & 2,400 s & 1,900.8 W.s \\
        & Relay Board & 12 V & 0.05 A & 0.6 W & 2,400 s & 1,440 W.s \\
        & Hydraulic Pump & 12 V & 0.7 A & 8.4 W & 2,400 s & 20,160 W.s \\
        Total & & & & & & 23,500.8 W.s \\
        \bottomrule
    \end{tabularx}
    \captionof{table}{Total Consumption during Pumping Water to the Tank.}
    \label{tbl:worstPumpingWatt}
\end{center}

NOTE: Power Consumed by the \textbf{Sensor} and the other \textbf{Relay Board} during their standby mode is not included in this table.

\subsubsection{Per Day Consumption}

\paragraph{Standby}

\begin{center}
    \begin{tabularx} {\textwidth} {
            >{\raggedright\arraybackslash\hsize=0.1\hsize}X
            >{\raggedright\arraybackslash\hsize=0.2\hsize}X
            *{4}{>{\centering\arraybackslash\hsize=0.125\hsize}X}
            >{\centering\arraybackslash\hsize=0.2\hsize}X
        }
        \toprule
        & {\bfseries Active Components} & {\bfseries Working Voltage (V)}
        & {\bfseries Current Draw (A)} & {\bfseries Power (W)}
        & {\bfseries Time Duration (s)} & {\bfseries Power x Time (W.s)} \\
        \midrule
        & Esp 32 & 3.3 V & 0.02 A & 0.066 W & 86,400 s & 5,702.4 W.s \\
        & Relay Board I & 12 V & 0.002 A & 0.024 W & 86,400 s & 2,073.6 W.s \\
        & Relay Board II & 12 V & 0.002 A & 0.024 W & 86,400 s & 2,073.6 W.s \\
        & DHT 22 Sensor & 5 V & 0.0005 A & 0.0025 W & 86,400 s & 216 W.s \\
        Total & & & & & & 10,065.6 W.s \\
        \bottomrule
    \end{tabularx}
    \captionof{table}{Total Consumption during the Standby Time for a day.}
    \label{tbl:worstStandbyWatt}
\end{center}


\paragraph{Sensing}

\begin{center}
    \begin{tabularx} {\textwidth} {
            >{\raggedright\arraybackslash\hsize=0.1\hsize}X
            >{\raggedright\arraybackslash\hsize=0.2\hsize}X
            *{4}{>{\centering\arraybackslash\hsize=0.125\hsize}X}
            >{\centering\arraybackslash\hsize=0.2\hsize}X
        }
        \toprule
        & {\bfseries Active Components} & {\bfseries Working Voltage (V)}
        & {\bfseries Current Draw (A)} & {\bfseries Power (W)}
        & {\bfseries Time Duration (s)} & {\bfseries Power x Time (W.s)} \\
        \midrule
        & Esp 32 & 3.3 V & 0.24 A & 0.792 W & 480 s & 380.16 W.s \\
        & DHT 22 Sensor & 5 V & 0.002 A & 0.01 W & 480 s & 4.8 W.s \\
        Total & & & & & & 384.96 W.s \\
        \bottomrule
    \end{tabularx}
    \captionof{table}{Total Consumption during the Sensing Time for a day.}
    \label{tbl:worstSensingWatt}
\end{center}

\paragraph{Watering}

\begin{center}
    \begin{tabularx} {\textwidth} {
            >{\raggedright\arraybackslash\hsize=0.1\hsize}X
            >{\raggedright\arraybackslash\hsize=0.2\hsize}X
            *{4}{>{\centering\arraybackslash\hsize=0.125\hsize}X}
            >{\centering\arraybackslash\hsize=0.2\hsize}X
        }
        \toprule
        & {\bfseries Active Components} & {\bfseries Working Voltage (V)}
        & {\bfseries Current Draw (A)} & {\bfseries Power (W)}
        & {\bfseries Time Duration (s)} & {\bfseries Power x Time (W.s)} \\
        \midrule
        & Esp 32 & 3.3 V & 0.24 A & 0.792 W & 160 s & 126.72 W.s \\
        & Solenoid Valve & 12 V & 0.6 A & 7.2 W & 160 s & 1,152 W.s \\
        & Relay Board & 12 V & 0.05 A & 0.6 W & 160 s & 96 W.s \\
        Total & & & & & & 1,374.72 W.s \\
        \bottomrule
    \end{tabularx}
    \captionof{table}{Total Consumption during the Watering Time for a day.}
    \label{tbl:worstWateringWatt}
\end{center}

\paragraph{Final Worst Case Per Day Consumption}

\begin{align*}
    \mbox{Final Worst Case Per Day Consumption} &= \mbox{Standby} + \mbox{Sensing} + \mbox{Watering} \\
    &= 10,065.6 + 384.96 + 1,374.72 \\
    &= 11,825.3 \: \mbox{W} \cdot \mbox{s}
\end{align*}

\subsubsection{Final Worst Case Per Cycle Consumption}

\begin{align*}
    \mbox{Final Worst Case Per Cycle Consumption} &= \mbox{Pumping} + (10 \times \mbox{Per Day}) \\
    &= 23,500.8 + (10 \times 11,825.3) \\
    &= 23,500.8 + 118,253 \\
    &= 141,754 \: \mbox{W} \cdot \mbox{s}
\end{align*}

\subsubsection{Backup available from a 12 V 1 AH Battery}

\begin{align*}
    \mbox{Total Watt Hours, battery can output} &= 12 \times 1 \\
    &= 12 \: \mbox{W} \cdot \mbox{H} \\ \\
    \mbox{Total Watt seconds, battery can output} &= \mbox{Total Watt Hours} \times 60 \times 60 \\
    &= 12 \times 60 \times 60 \\
    &= 43,200 \: \mbox{W} \cdot \mbox{s}
\end{align*}

Clearly the per cycle demand, \textbf{141,754} $\mbox{W} \cdot \mbox{s}$
is much \textbf{larger} than what this battery can provide, \textbf{43,200}
$\mbox{W} \cdot \mbox{s}$. An approximate amount of backup will be,

\begin{align*}
    \mbox{Pumping water is inevitable in a single Duty Cycle.} \\
    \therefore \mbox{Watt seconds available for the rest of the Cycle} &= \mbox{Total Output} - \mbox{Pumping Water} \\
    &= 43,200 - 23,500.8 \\
    &= 19,699.2 \: \mbox{W} \cdot \mbox{s}
\end{align*}

\begin{align*}
    \mbox{Total days this battery can last} &= \dfrac{\mbox{Rest of the Energy Available}}{\mbox{Per Day Consumption}} \\
    &= \dfrac{19,699.2}{11,825.3} \\
    &= 1.67 \: \mbox{Days} \: (\approx 1 \: \mbox{Days and} \: 16 \: \mbox{Hours})
\end{align*}

\subsubsection{Backup available from a 12 V 5 AH Battery}

\begin{align*}
    \mbox{Total Watt Hours, battery can output} &= 12 \times 5 \\
    &= 60 \: \mbox{W} \cdot \mbox{H} \\ \\
    \mbox{Total Watt seconds, battery can output} &= \mbox{Total Watt Hours} \times 60 \times 60 \\
    &= 60 \times 60 \times 60 \\
    &= 216,000 \: \mbox{W} \cdot \mbox{s}
\end{align*}

As we can see the per cycle demand, \textbf{141,754} $\mbox{W} \cdot \mbox{s}$
is \textbf{lesser} than what this battery can provide, \textbf{216,000}
$\mbox{W} \cdot \mbox{s}$. An approximate amount of backup will be,

\begin{align*}
    \mbox{Pumping water is inevitable in a single Duty Cycle.} \\
    \therefore \mbox{Watt seconds available for the rest of the Cycle} &= \mbox{Total Output} - \mbox{Pumping Water} \\
    &= 216,000 - 23,500.8 \\
    &= 192,499 \: \mbox{W} \cdot \mbox{s}
\end{align*}

\begin{align*}
    \mbox{Total days this battery can last} &= \dfrac{\mbox{Rest of the Energy Available}}{\mbox{Per Day Consumption}} \\
    &= \dfrac{192,499}{11,825.3} \\
    &= 16.28 \: \mbox{Days} \: (\approx 16 \: \mbox{Days and} \: 6 \: \mbox{Hours})
\end{align*}


\subsection{Ideal Case Scenario}

\subsubsection{Assumptions}

In the ideal case scenario, we are assuming these things to be true:

\begin{itemize}
    \item Esp 32 is in \textbf{Active Mode II}, while we are doing something, and in \textbf{Sleep Mode}, while Standby.
\end{itemize}

\subsubsection{Pumping Water}

\begin{center}
    \begin{tabularx} {\textwidth} {
            >{\raggedright\arraybackslash\hsize=0.1\hsize}X
            >{\raggedright\arraybackslash\hsize=0.2\hsize}X
            *{4}{>{\centering\arraybackslash\hsize=0.125\hsize}X}
            >{\centering\arraybackslash\hsize=0.2\hsize}X
        }
        \toprule
        & {\bfseries Active Components} & {\bfseries Working Voltage (V)}
        & {\bfseries Current Draw (A)} & {\bfseries Power (W)}
        & {\bfseries Time Duration (s)} & {\bfseries Power x Time (W.s)} \\
        \midrule
        & Esp 32 & 3.3 V & 0.16 A & 0.528 W & 2,400 s & 1,267.2 W.s \\
        & Relay Board & 12 V & 0.05 A & 0.6 W & 2,400 s & 1,440 W.s \\
        & Hydraulic Pump & 12 V & 0.7 A & 8.4 W & 2,400 s & 20,160 W.s \\
        Total & & & & & & 22,867.2 W.s \\
        \bottomrule
    \end{tabularx}
    \captionof{table}{Total Consumption during Pumping Water to the Tank.}
    \label{tbl:idealPumpingWatt}
\end{center}

NOTE: Power Consumed by the \textbf{Sensor} and the other \textbf{Relay Board} during their standby mode is not included in this table.

\subsubsection{Per Day Consumption}

\paragraph{Standby}

\begin{center}
    \begin{tabularx} {\textwidth} {
            >{\raggedright\arraybackslash\hsize=0.1\hsize}X
            >{\raggedright\arraybackslash\hsize=0.2\hsize}X
            *{4}{>{\centering\arraybackslash\hsize=0.125\hsize}X}
            >{\centering\arraybackslash\hsize=0.2\hsize}X
        }
        \toprule
        & {\bfseries Active Components} & {\bfseries Working Voltage (V)}
        & {\bfseries Current Draw (A)} & {\bfseries Power (W)}
        & {\bfseries Time Duration (s)} & {\bfseries Power x Time (W.s)} \\
        \midrule
        & Esp 32 & 3.3 V & 0.001 A & 0.0033 W & 86,400 s & 285.12 W.s \\
        & Relay Board I & 12 V & 0.002 A & 0.024 W & 86,400 s & 2,073.6 W.s \\
        & Relay Board II & 12 V & 0.002 A & 0.024 W & 86,400 s & 2,073.6 W.s \\
        & DHT 22 Sensor & 5 V & 0.0005 A & 0.0025 W & 86,400 s & 216 W.s \\
        Total & & & & & & 4,648.32 W.s \\
        \bottomrule
    \end{tabularx}
    \captionof{table}{Total Consumption during the Standby Time for a day.}
    \label{tbl:idealStandbyWatt}
\end{center}

\paragraph{Sensing}

\begin{center}
    \begin{tabularx} {\textwidth} {
            >{\raggedright\arraybackslash\hsize=0.1\hsize}X
            >{\raggedright\arraybackslash\hsize=0.2\hsize}X
            *{4}{>{\centering\arraybackslash\hsize=0.125\hsize}X}
            >{\centering\arraybackslash\hsize=0.2\hsize}X
        }
        \toprule
        & {\bfseries Active Components} & {\bfseries Working Voltage (V)}
        & {\bfseries Current Draw (A)} & {\bfseries Power (W)}
        & {\bfseries Time Duration (s)} & {\bfseries Power x Time (W.s)} \\
        \midrule
        & Esp 32 & 3.3 V & 0.16 A & 0.528 W & 480 s & 253.44 W.s \\
        & DHT 22 Sensor & 5 V & 0.002 A & 0.01 W & 480 s & 4.8 W.s \\
        Total & & & & & & 258.24 W.s \\
        \bottomrule
    \end{tabularx}
    \captionof{table}{Total Consumption during the Sensing Time for a day.}
    \label{tbl:idealSensingWatt}
\end{center}

\paragraph{Watering}

\begin{center}
    \begin{tabularx} {\textwidth} {
            >{\raggedright\arraybackslash\hsize=0.1\hsize}X
            >{\raggedright\arraybackslash\hsize=0.2\hsize}X
            *{4}{>{\centering\arraybackslash\hsize=0.125\hsize}X}
            >{\centering\arraybackslash\hsize=0.2\hsize}X
        }
        \toprule
        & {\bfseries Active Components} & {\bfseries Working Voltage (V)}
        & {\bfseries Current Draw (A)} & {\bfseries Power (W)}
        & {\bfseries Time Duration (s)} & {\bfseries Power x Time (W.s)} \\
        \midrule
        & Esp 32 & 3.3 V & 0.16 A & 0.528 W & 160 s & 84.48 W.s \\
        & Solenoid Valve & 12 V & 0.6 A & 7.2 W & 160 s & 1,152 W.s \\
        & Relay Board & 12 V & 0.05 A & 0.6 W & 160 s & 96 W.s \\
        Total & & & & & & 1,332.48 W.s \\
        \bottomrule
    \end{tabularx}
    \captionof{table}{Total Consumption during the Watering Time for a day.}
    \label{tbl:idealWateringWatt}
\end{center}


\paragraph{Final Ideal Case Per Day Consumption}

\begin{align*}
    \mbox{Final Ideal Case Per Day Consumption} &= \mbox{Standby} + \mbox{Sensing} + \mbox{Watering} \\
    &= 4,648.32 + 258.24 + 1,332.48 \\
    &= 6,239.04 \: \mbox{W} \cdot \mbox{s}
\end{align*}

\subsubsection{Final Ideal Case Per Cycle Consumption}

\begin{align*}
    \mbox{Final Ideal Case Per Cycle Consumption} &= \mbox{Pumping} + (10 \times \mbox{Per Day}) \\
    &= 22,867.2 + (10 \times 6,239.04) \\
    &= 22,867.2 + 62,390.4 \\
    &= 85,257.6 \: \mbox{W} \cdot \mbox{s}
\end{align*}

\subsubsection{Backup available from a 12 V 1 AH Battery}

\begin{align*}
    \mbox{Total Watt Hours, battery can output} &= 12 \times 1 \\
    &= 12 \: \mbox{W} \cdot \mbox{H} \\ \\
    \mbox{Total Watt seconds, battery can output} &= \mbox{Total Watt Hours} \times 60 \times 60 \\
    &= 12 \times 60 \times 60 \\
    &= 43,200 \: \mbox{W} \cdot \mbox{s}
\end{align*}

The per cycle demand, \textbf{85,257.6} $\mbox{W} \cdot \mbox{s}$ is
\textbf{double} than that of the battery can provide, \textbf{43,200} $\mbox{W}
\cdot \mbox{s}$. An approximate amount of backup will be,

\begin{align*}
    \mbox{Pumping water is inevitable in a single Duty Cycle.} \\
    \therefore \mbox{Watt seconds available for the rest of the Cycle} &= \mbox{Total Output} - \mbox{Pumping Water} \\
    &= 43,200 - 22,867.2 \\
    &= 20,332.8 \: \mbox{W} \cdot \mbox{s}
\end{align*}

\begin{align*}
    \mbox{Total days this battery can last} &= \dfrac{\mbox{Rest of the Energy Available}}{\mbox{Per Day Consumption}} \\
    &= \dfrac{20,332.8}{6,239.04} \\
    &= 3.25 \: \mbox{Days} \: (\approx 3 \: \mbox{Days and} \: 6 \: \mbox{Hours})
\end{align*}

\subsubsection{Backup available from a 12 V 5 AH Battery}

\begin{align*}
    \mbox{Total Watt Hours, battery can output} &= 12 \times 5 \\
    &= 60 \: \mbox{W} \cdot \mbox{H} \\ \\
    \mbox{Total Watt seconds, battery can output} &= \mbox{Total Watt Hours} \times 60 \times 60 \\
    &= 60 \times 60 \times 60 \\
    &= 216,000 \: \mbox{W} \cdot \mbox{s}
\end{align*}

Clearly the per cycle demand, \textbf{85,257.6} $\mbox{W} \cdot \mbox{s}$ is
much \textbf{less} than what this battery can provide, \textbf{216,000}
$\mbox{W} \cdot \mbox{s}$. An approximate amount of backup will be,

\begin{align*}
    \mbox{Pumping water is inevitable in a single Duty Cycle.} \\
    \therefore \mbox{Watt seconds available for the rest of the Cycle} &= \mbox{Total Output} - \mbox{Pumping Water} \\
    &= 216,000 - 22,867.2 \\
    &= 193,133 \: \mbox{W} \cdot \mbox{s}
\end{align*}

\begin{align*}
    \mbox{Total days this battery can last} &= \dfrac{\mbox{Rest of the Energy Available}}{\mbox{Per Day Consumption}} \\
    &= \dfrac{193,133}{6,239.04} \\
    &= 30.9 \: \mbox{Days} \: (\approx 30 \: \mbox{Days and} \: 21 \: \mbox{Hours})
\end{align*}

\end{document}
